\documentclass[10pt]{article}
\usepackage{quiver}
%%%%%%%%%%%%%%%%%%%%%%%%%%%%%%%%%%%%%%%%%
% Specification file for
% Custom Assignment Template (For Students) by Paul Zhang
%
% Based on:
% Cleese Assignment (For Students) Version 2.0 (27/5/2018)
%
% This template originates from http://www.LaTeXTemplates.com
% Original Author: Vel (vel@LaTeXTemplates.com)
%
% License:
% CC BY-NC-SA 3.0 (http://creativecommons.org/licenses/by-nc-sa/3.0/)
%
%%%%%%%%%%%%%%%%%%%%%%%%%%%%%%%%%%%%%%%%%

%----------------------------------------------------------------------------------------
%	PACKAGES AND OTHER DOCUMENT CONFIGURATIONS
%----------------------------------------------------------------------------------------

\usepackage{
amsfonts,     % Extra fonts
amsmath,      % Lots of useful mathematical symbols
amssymb,      % More useful mathematical symbols
amsthm,       % Theorem and proof environments
cite,         % Citations
enumitem,     % Change enumerate counter style
fancyhdr, 	  % Fancy header
float,		  % Figure placement
geometry,     % Required for adjusting page dimensions and margins
graphicx,     % Required to insert images
hyperref,     % Hyperlinks
ifthen,       % Required for ifthenelse statements
import,       % So that tex files can input from relative paths
listings,     % Required for insertion of code
mathtools,    % Extra math symbols
multicol,     % Easy setup for local multi-column use
pgfplots,     % For drawing graphs of functions
tikz,         % Nice diagrams
xcolor,       % Used for changing text color
}

\pgfplotsset{compat=1.17}
\hypersetup{
  colorlinks   = true,    % Colours links instead of ugly boxes
  urlcolor     = blue,    % Colour for external hyperlinks
  linkcolor    = blue,    % Colour of internal links
  citecolor    = red      % Colour of citations
}

\usepackage[most]{tcolorbox} % Required for boxes that split across pages
\usepackage[justification=centering, margin=1.3cm]{caption} % Center long captions

\setlength{\parindent}{0pt}

%----------------------------------------------------------------------------------------
%	LISTINGS
%----------------------------------------------------------------------------------------

\definecolor{codegreen}{rgb}{0,0.6,0}
\definecolor{codegray}{rgb}{0.5,0.5,0.5}
\definecolor{codepurple}{rgb}{0.58,0,0.82}
\definecolor{backcolour}{rgb}{0.95,0.95,0.92}

\lstdefinestyle{mystyle}{
	backgroundcolor=\color{backcolour},
	commentstyle=\color{codegreen},
	keywordstyle=\color{magenta},
	numberstyle=\tiny\color{codegray},
	stringstyle=\color{codepurple},
	basicstyle=\ttfamily,
	breakatwhitespace=false,
	breaklines=true,
	captionpos=b,
	keepspaces=true,
	numbers=left,
	numbersep=5pt,
	showspaces=false,
	showstringspaces=false,
	showtabs=false,
	tabsize=2
}

\lstdefinestyle{mystyletiny}{
	backgroundcolor=\color{backcolour},
	commentstyle=\color{codegreen},
	keywordstyle=\color{magenta},
	numberstyle=\tiny\color{codegray},
	stringstyle=\color{codepurple},
	basicstyle=\tiny\ttfamily,
	breakatwhitespace=false,
	breaklines=true,
	captionpos=b,
	keepspaces=true,
	numbers=left,
	numbersep=5pt,
	showspaces=false,
	showstringspaces=false,
	showtabs=false,
	tabsize=2
}

\lstset{style=mystyle}

%----------------------------------------------------------------------------------------
%	MARGINS
%----------------------------------------------------------------------------------------

\usepackage{geometry} % Required for adjusting page dimensions and margins

\geometry{
	paper=a4paper, % Change to letterpaper for US letter
	top=1.7cm, % Top margin
	bottom=1.7cm, % Bottom margin
	left=2.5cm, % Left margin
	right=2.5cm, % Right margin
	headheight=14pt, % Header height
	footskip=0.5cm, % Space from the bottom margin to the baseline of the footer
	headsep=0.5cm, % Space from the top margin to the baseline of the header
	%showframe, % Uncomment to show how the type block is set on the page
}

%----------------------------------------------------------------------------------------
%	FONT
%----------------------------------------------------------------------------------------

\usepackage[utf8]{inputenc} % Required for inputting international characters
\usepackage[T1]{fontenc} % Output font encoding for international characters

\usepackage{tgpagella} % Use the TEX Gyre Pagella font

%----------------------------------------------------------------------------------------
%	HEADERS AND FOOTERS
%----------------------------------------------------------------------------------------

\usepackage{fancyhdr} % Required for customising headers and footers

\pagestyle{fancy} % Enable custom headers and footers

\lhead{} % Left header
\chead{} % Centre header
\rhead{}

\lfoot{} % Left footer
% \cfoot{} % Centre footer, shows page numbers
\rfoot{} % Right footer

\renewcommand\headrulewidth{0pt} % Remove header rule

%----------------------------------------------------------------------------------------
%	MODIFY SECTION STYLES
%----------------------------------------------------------------------------------------

\usepackage{titlesec} % Required for modifying sections

\pagenumbering{arabic}

%------------------------------------------------
% Section

\titleformat
{\section} % Section type being modified
[block] % Shape type, can be: hang, block, display, runin, leftmargin, rightmargin, drop, wrap, frame
{\LARGE\bfseries} % Format of the whole section
{\assignmentQuestionName~\thesection} % Format of the section label
{6pt} % Space between the title and label
{} % Code before the label

\titlespacing{\section}{0pt}{0.5\baselineskip}{0.5\baselineskip} % Spacing around section titles, the order is: left, before and after

%------------------------------------------------
% Subsection

\titleformat
{\subsection} % Section type being modified
[block] % Shape type, can be: hang, block, display, runin, leftmargin, rightmargin, drop, wrap, frame
{\large\bfseries} % Format of the whole section
{(\alph{subsection})} % Format of the section label
{4pt} % Space between the title and label
{} % Code before the label

\titlespacing{\subsection}{0pt}{0.5\baselineskip}{0.5\baselineskip} % Spacing around section titles, the order is: left, before and after

\renewcommand\thesubsection{(\alph{subsection})}

%------------------------------------------------
% Theorem, Lemma, etc

\newenvironment{question}[1][]{
    {\noindent \textbf{\ifthenelse{\equal{#1}{}}{Question. }{Question [#1]. }}}}
    {}

\newenvironment{definition}[1][]{
    {\noindent \textbf{\ifthenelse{\equal{#1}{}}{Definition. }{Definition [#1]. }}}}
    {}

\newenvironment{solution}{
    ~\\ \noindent \textbf{My solution.} \par}
    {}

\newenvironment{theorem}[1][]{
    {\noindent \textbf{\ifthenelse{\equal{#1}{}}{Theorem. }{Theorem [#1]. }}}}
    {}

\newenvironment{lemma}[1][]{
    {\noindent \textbf{\ifthenelse{\equal{#1}{}}{Lemma. }{Lemma [#1]. }}}}
    {}

\newenvironment{proposition}[1][]{
    {\noindent \textbf{\ifthenelse{\equal{#1}{}}{Proposition. }{Proposition [#1]. }}}}
    {}

\renewenvironment{comment}{
    {\noindent \textbf{Comment. }}}
    {}


%------------------------------------------------
% Math Commands

\newcommand{\N}{\mathbb N}        % Naturals
\newcommand{\Z}{\mathbb Z}        % Integers
\newcommand{\Q}{\mathbb Q}        % Rationals
\newcommand{\R}{\mathbb R}        % Reals
\newcommand{\mb}{\mathbf}
\newcommand{\mc}{\mathcal}
\newcommand{\mf}{\mathfrak}
\newcommand{\inver}{^{-1}}
\newcommand{\floor}[1]{\lfloor #1 \rfloor}
\newcommand{\ceil}[1]{\lceil #1 \rceil}

%------------------------------------------------
% Tikz settings
\usetikzlibrary{shapes.geometric}
\usetikzlibrary{arrows.meta}
\usetikzlibrary{automata, calc, positioning}

\usepackage{xparse}% http://ctan.org/pkg/xparse
\usepackage{etoolbox}% http://ctan.org/pkg/etoolbox
\newcounter{listtotal}\newcounter{listcntr}%
\NewDocumentCommand{\entries}{o}{%
	\setcounter{listtotal}{0}\setcounter{listcntr}{-1}%
	\renewcommand*{\do}[1]{\stepcounter{listtotal}}%
	\expandafter\docsvlist\expandafter{\entriesarray}%
	\IfNoValueTF{#1}
	{\entriesarray}% \names
	{% \names[<index>]
		\renewcommand*{\do}[1]{\stepcounter{listcntr}\ifnum\value{listcntr}=#1\relax##1\fi}%
		\expandafter\docsvlist\expandafter{\entriesarray}}%
}
\newcommand{\entriesarray}{}


\lhead{}
\rhead{CSC363H5 Winter 2023} 
\chead{\textbf{Tutorial 10: NP-Completeness II}}
\date{}
\begin{document}
		
\subsection*{Introduction}

\begin{definition}[\textsf{3SAT}]
	We say a formula \(\varphi\) is in \textit{conjunctive normal form}, or \(\varphi\) is a cnf-formula, if it can be written as a conjunction (\(\land\)) of \textit{clauses}, and each clause is a disjunction (\(\lor\)) of literals (either a variable \(x_i\) or its negation  \(\lnot x_i\)). 
\begin{align*}
	\varphi = \bigwedge_{i=1}^n\left(\bigvee_{j=1}^{k_i} f_{ij}\right) =  (f_{11} \lor \dots \lor f_{1k_1}) \land \dots \land (f_{n1} \lor \dots \lor f_{nk_n}).
\end{align*}
A \textit{3cnf-formula} is a cnf-formula such that each \textit{clause} has exactly \textit{three literals}.  For example:
\begin{align*}
	(x_1 \lor x_3 \lor \lnot x_2) \land (x_1 \lor \lnot x_1 \lor x_4) \land (\lnot x_5 \lor \lnot x_2 \lor \lnot x_7)& \text{ is a 3cnf-formula.}\\
	(x_1 \lor x_3 \lor \lnot x_2) \land (x_1)\text{ and }
	\lnot(x_4 \land x_1 \lor x_3) \land  (x_1 \lor \lnot x_1 \land x_4) & \text{ are not 3cnf-formulas.}
\end{align*}
Finally, we can define:
\begin{align*}
	\textsf{3SAT} = \{\langle\varphi\rangle: \varphi \text{ is a satisfiable 3cnf-formula}\} \subseteq \textsf{SAT}.
\end{align*}
\end{definition}
The following theorem is striking, since it shows that increasing the \textit{vertical complexity} of a formula does not necessarily allow it to be more expressive.\\

\begin{theorem}
	\textsf{3SAT} is NP-complete.
\end{theorem}
\\\\
If this theorem feels ``obvious'', then it may be more surprising that \(\textsf{2SAT}\) is not even NP-complete! This indicates that there is a \textit{fundamental difference in the difficulty} of solving cnf-formulas with clauses of size at most 2, and cnf-formulas with clauses of size at least 3. 
\subsection*{Part 1.}

In this tutorial, we will study the problem \textsf{VERTEX-COVER}, and eventually prove that it is NP-complete. \textsf{VERTEX-COVER} is phrased as follows: Given a graph \(G\) and \(k \in \N\), does there exist a set of  \(k\) vertices \(v_1, \dots, v_k\) in \(G\), such that every edge in \(G\) is incident with at least one of the \(v_i\)'s? If so, we say that \(\{v_1, \dots, v_k\}\) is a \(k\)-(vertex) cover of \(G\).\\
\begin{align*}
	\textsf{VERTEX-COVER} &= \{\langle G,k \rangle: G \text{ has a \(k\)-cover}\}
\end{align*}

\textbf{Exercise 1.} Try to find a \(3\)-cover of the following graph:
\begin{center}
\begin{tikzcd}
	\bullet && \bullet \\
	\bullet & \bullet && \bullet \\
	& \bullet & \bullet
	\arrow[no head, from=1-1, to=1-3]
	\arrow[no head, from=1-3, to=2-1]
	\arrow[no head, from=1-1, to=2-1]
	\arrow[no head, from=1-1, to=2-4]
	\arrow[no head, from=2-1, to=3-2]
	\arrow[no head, from=2-4, to=3-3]
	\arrow[no head, from=1-3, to=3-3]
	\arrow[no head, from=2-2, to=3-3]
\end{tikzcd}
\end{center}
\textbf{Exercise 2.} Reason that \( \textsf{VERTEX-COVER} \in \text{NP}\) using one of the following two methods:
\begin{enumerate}
	\item Show that there is a polynomial time verifier \(V(\langle G, k\rangle, c)\) for \( \textsf{VERTEX-COVER}\). \textit{Hint: the input \(c\) codes a ``solution'' to the problem; the format of \(c\) depends on the definition of \(V\) and should be chosen by you. What counts as evidence that \(G\) has a \(k\)-cover?}
	\item Show that there is a polynomial time NTM \(T(\langle G, k\rangle)\) that decides \(\textsf{VERTEX-COVER}\). \textit{Hint: at some point in its computation, \(T\) will have to nondeterministically guess a property of \(G\), and later it should \textbf{deterministically} check that the guess was a correct one.}
\end{enumerate}

\subsection*{Part 2}

We now show that \(\textsf{VERTEX-COVER}\) is NP-hard, hence proving that it is NP-complete. As a sanity check, recall that it suffices to show
\begin{align*}
	\textsf{3SAT} \leq_{\text{P}} \textsf{VERTEX-COVER}.
\end{align*}
In other words, given any 3cnf-formula \(\varphi(x_1, \dots, x_n) = \varphi_1(\vec{x}) \land \dots \land \varphi_\ell(\vec{x})\), we will construct in polynomial time (of \(|\varphi|\)) a graph \(\Gamma \varphi\) and an integer \(k\), such that \(\Gamma\varphi\) has a  \(k\)-clique iff \(\varphi\) is satisfiable.\\

Let \(\Gamma \varphi\) be the graph constructed by the following steps:
\begin{enumerate}
	\item For each variable \(x_i\) in \(\varphi\), \(\Gamma \varphi\) will have the subgraph
\begin{center}
\begin{tikzcd}
	{x_i} & {\lnot x_i}
	\arrow[no head, from=1-1, to=1-2]
\end{tikzcd}	
\end{center}
\item For each clause \(\varphi_j = (y_j \lor z_j \lor w_j)\), where, for example, \(y_j\) is a literal of the form \(x_i\) or \(\lnot x_i\) for some  \(i\), \(\Gamma \varphi\) will have the subgraph
\begin{center}
\begin{tikzcd}
	{y_i} & {z_j} \\
	& {w_j}
	\arrow[no head, from=1-1, to=1-2]
	\arrow[no head, from=2-2, to=1-1]
	\arrow[no head, from=1-2, to=2-2]
\end{tikzcd}	
\end{center}
\item Finally, connect each literal \(y_j, z_j, w_j\) to the corresponding variable in part 1, respecting negation.
\end{enumerate}

For example, the formula \(\varphi(x_1, x_2) = (x_1 \lor \lnot x_2 \lor x_1) \land (\lnot x_2\lor  \lnot x_1 \lor x_2)\) will have the graph \(\Gamma \varphi\) given by
\begin{center}
\begin{tikzcd}
	{x_1} & {\lnot x_1} & {x_2} & {\lnot x_2} \\
	{x_1} & {\lnot x_2} & {\lnot x_2} & {\lnot x_1} \\
	& {x_1} && {x_2}
	\arrow[no head, from=2-1, to=3-2]
	\arrow[no head, from=2-2, to=3-2]
	\arrow[no head, from=2-1, to=2-2]
	\arrow[no head, from=2-3, to=3-4]
	\arrow[no head, from=2-4, to=3-4]
	\arrow[no head, from=2-3, to=2-4]
	\arrow[dashed, no head, from=3-2, to=1-1]
	\arrow[dashed, no head, from=2-1, to=1-1]
	\arrow[no head, from=1-1, to=1-2]
	\arrow[no head, from=1-3, to=1-4]
	\arrow[dashed, no head, from=1-4, to=2-3]
	\arrow[dashed, no head, from=3-4, to=1-3]
	\arrow[dashed, no head, from=2-4, to=1-2]
	\arrow[dashed, no head, from=2-2, to=1-4]
\end{tikzcd}	
\end{center}

\textbf{Exercise 3.} Practice constructing the following graph:
\begin{align*}
	\Gamma((x_1 \lor \lnot x_2 \lor \lnot x_3 ) \land (\lnot x_2 \lor \lnot x_1 \lor x_1 ) \land (x_3 \lor x_2 \lor x_1))
\end{align*}
\textbf{Exercise 4.} Suppose \(\varphi\) is satisfiable, meaning we have an assignment  \(f: \{x_1, \dots, x_n\} \to \{0,1\}\). We construct a cover \(\mathcal{C}\) of \(\Gamma \varphi\) as follows: 
\begin{enumerate}
\item \(\mathcal{C}\) will contain exactly one of \(x_i\) and \(\lnot x_i\) corresponding to whether \(f(x_i)\) equals  \(1\) or \(0\), respectively.
\item Since each clause \(\varphi_j = (y_j \lor z_j \lor w_j)\) is satisfied, then at least one of the literals of \(\varphi_j\) equals 1. Let \(\mathcal{C}\) contain the vertices corresponding to the other two literals.
\end{enumerate}
Using the solution \(f(x_1) = 1, f(x_2) = 0, f(x_3) = 0\) for \(\varphi\) in exercise 3, find the cover \(\mathcal{C}\).\\

\textbf{Exercise 5.} Prove in general that \(\mathcal{C}\) is a \(k\)-cover for \(\Gamma \varphi\), where \(k = n + 2\ell\), \(n\) is the number of variables, and \(\ell\) is the number of clauses.\\

\textbf{Exercise 6.} Now work backwards: if \(\Gamma \varphi\) has a  \(k\)-cover \(\mathcal{C}\), where \(k = n + 2\ell\), prove that  \(\varphi\) is satisfiable. \textit{Hint: first, reason that at least one vertex from each pair \(x_i, \lnot x_i\) is in  \(\mathcal{C}\), and similarly that \(\mathcal{C}\)  must contain vertices corresponding to at least two literals from each clause. Finally, conclude that this  choice of vertices actually represents a solution to \(\varphi\).}
\end{document}
