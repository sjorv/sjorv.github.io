\documentclass[10pt]{article}
%%%%%%%%%%%%%%%%%%%%%%%%%%%%%%%%%%%%%%%%%
% Specification file for
% Custom Assignment Template (For Students) by Paul Zhang
%
% Based on:
% Cleese Assignment (For Students) Version 2.0 (27/5/2018)
%
% This template originates from http://www.LaTeXTemplates.com
% Original Author: Vel (vel@LaTeXTemplates.com)
%
% License:
% CC BY-NC-SA 3.0 (http://creativecommons.org/licenses/by-nc-sa/3.0/)
%
%%%%%%%%%%%%%%%%%%%%%%%%%%%%%%%%%%%%%%%%%

%----------------------------------------------------------------------------------------
%	PACKAGES AND OTHER DOCUMENT CONFIGURATIONS
%----------------------------------------------------------------------------------------

\usepackage{
amsfonts,     % Extra fonts
amsmath,      % Lots of useful mathematical symbols
amssymb,      % More useful mathematical symbols
amsthm,       % Theorem and proof environments
cite,         % Citations
enumitem,     % Change enumerate counter style
fancyhdr, 	  % Fancy header
float,		  % Figure placement
geometry,     % Required for adjusting page dimensions and margins
graphicx,     % Required to insert images
hyperref,     % Hyperlinks
ifthen,       % Required for ifthenelse statements
import,       % So that tex files can input from relative paths
listings,     % Required for insertion of code
mathtools,    % Extra math symbols
multicol,     % Easy setup for local multi-column use
pgfplots,     % For drawing graphs of functions
tikz,         % Nice diagrams
xcolor,       % Used for changing text color
}

\pgfplotsset{compat=1.17}
\hypersetup{
  colorlinks   = true,    % Colours links instead of ugly boxes
  urlcolor     = blue,    % Colour for external hyperlinks
  linkcolor    = blue,    % Colour of internal links
  citecolor    = red      % Colour of citations
}

\usepackage[most]{tcolorbox} % Required for boxes that split across pages
\usepackage[justification=centering, margin=1.3cm]{caption} % Center long captions

\setlength{\parindent}{0pt}

%----------------------------------------------------------------------------------------
%	LISTINGS
%----------------------------------------------------------------------------------------

\definecolor{codegreen}{rgb}{0,0.6,0}
\definecolor{codegray}{rgb}{0.5,0.5,0.5}
\definecolor{codepurple}{rgb}{0.58,0,0.82}
\definecolor{backcolour}{rgb}{0.95,0.95,0.92}

\lstdefinestyle{mystyle}{
	backgroundcolor=\color{backcolour},
	commentstyle=\color{codegreen},
	keywordstyle=\color{magenta},
	numberstyle=\tiny\color{codegray},
	stringstyle=\color{codepurple},
	basicstyle=\ttfamily,
	breakatwhitespace=false,
	breaklines=true,
	captionpos=b,
	keepspaces=true,
	numbers=left,
	numbersep=5pt,
	showspaces=false,
	showstringspaces=false,
	showtabs=false,
	tabsize=2
}

\lstdefinestyle{mystyletiny}{
	backgroundcolor=\color{backcolour},
	commentstyle=\color{codegreen},
	keywordstyle=\color{magenta},
	numberstyle=\tiny\color{codegray},
	stringstyle=\color{codepurple},
	basicstyle=\tiny\ttfamily,
	breakatwhitespace=false,
	breaklines=true,
	captionpos=b,
	keepspaces=true,
	numbers=left,
	numbersep=5pt,
	showspaces=false,
	showstringspaces=false,
	showtabs=false,
	tabsize=2
}

\lstset{style=mystyle}

%----------------------------------------------------------------------------------------
%	MARGINS
%----------------------------------------------------------------------------------------

\usepackage{geometry} % Required for adjusting page dimensions and margins

\geometry{
	paper=a4paper, % Change to letterpaper for US letter
	top=1.7cm, % Top margin
	bottom=1.7cm, % Bottom margin
	left=2.5cm, % Left margin
	right=2.5cm, % Right margin
	headheight=14pt, % Header height
	footskip=0.5cm, % Space from the bottom margin to the baseline of the footer
	headsep=0.5cm, % Space from the top margin to the baseline of the header
	%showframe, % Uncomment to show how the type block is set on the page
}

%----------------------------------------------------------------------------------------
%	FONT
%----------------------------------------------------------------------------------------

\usepackage[utf8]{inputenc} % Required for inputting international characters
\usepackage[T1]{fontenc} % Output font encoding for international characters

\usepackage{tgpagella} % Use the TEX Gyre Pagella font

%----------------------------------------------------------------------------------------
%	HEADERS AND FOOTERS
%----------------------------------------------------------------------------------------

\usepackage{fancyhdr} % Required for customising headers and footers

\pagestyle{fancy} % Enable custom headers and footers

\lhead{} % Left header
\chead{} % Centre header
\rhead{}

\lfoot{} % Left footer
% \cfoot{} % Centre footer, shows page numbers
\rfoot{} % Right footer

\renewcommand\headrulewidth{0pt} % Remove header rule

%----------------------------------------------------------------------------------------
%	MODIFY SECTION STYLES
%----------------------------------------------------------------------------------------

\usepackage{titlesec} % Required for modifying sections

\pagenumbering{arabic}

%------------------------------------------------
% Section

\titleformat
{\section} % Section type being modified
[block] % Shape type, can be: hang, block, display, runin, leftmargin, rightmargin, drop, wrap, frame
{\LARGE\bfseries} % Format of the whole section
{\assignmentQuestionName~\thesection} % Format of the section label
{6pt} % Space between the title and label
{} % Code before the label

\titlespacing{\section}{0pt}{0.5\baselineskip}{0.5\baselineskip} % Spacing around section titles, the order is: left, before and after

%------------------------------------------------
% Subsection

\titleformat
{\subsection} % Section type being modified
[block] % Shape type, can be: hang, block, display, runin, leftmargin, rightmargin, drop, wrap, frame
{\large\bfseries} % Format of the whole section
{(\alph{subsection})} % Format of the section label
{4pt} % Space between the title and label
{} % Code before the label

\titlespacing{\subsection}{0pt}{0.5\baselineskip}{0.5\baselineskip} % Spacing around section titles, the order is: left, before and after

\renewcommand\thesubsection{(\alph{subsection})}

%------------------------------------------------
% Theorem, Lemma, etc

\newenvironment{question}[1][]{
    {\noindent \textbf{\ifthenelse{\equal{#1}{}}{Question. }{Question [#1]. }}}}
    {}

\newenvironment{definition}[1][]{
    {\noindent \textbf{\ifthenelse{\equal{#1}{}}{Definition. }{Definition [#1]. }}}}
    {}

\newenvironment{solution}{
    ~\\ \noindent \textbf{My solution.} \par}
    {}

\newenvironment{theorem}[1][]{
    {\noindent \textbf{\ifthenelse{\equal{#1}{}}{Theorem. }{Theorem [#1]. }}}}
    {}

\newenvironment{lemma}[1][]{
    {\noindent \textbf{\ifthenelse{\equal{#1}{}}{Lemma. }{Lemma [#1]. }}}}
    {}

\newenvironment{proposition}[1][]{
    {\noindent \textbf{\ifthenelse{\equal{#1}{}}{Proposition. }{Proposition [#1]. }}}}
    {}

\renewenvironment{comment}{
    {\noindent \textbf{Comment. }}}
    {}


%------------------------------------------------
% Math Commands

\newcommand{\N}{\mathbb N}        % Naturals
\newcommand{\Z}{\mathbb Z}        % Integers
\newcommand{\Q}{\mathbb Q}        % Rationals
\newcommand{\R}{\mathbb R}        % Reals
\newcommand{\mb}{\mathbf}
\newcommand{\mc}{\mathcal}
\newcommand{\mf}{\mathfrak}
\newcommand{\inver}{^{-1}}
\newcommand{\floor}[1]{\lfloor #1 \rfloor}
\newcommand{\ceil}[1]{\lceil #1 \rceil}

%------------------------------------------------
% Tikz settings
\usetikzlibrary{shapes.geometric}
\usetikzlibrary{arrows.meta}
\usetikzlibrary{automata, calc, positioning}

\usepackage{xparse}% http://ctan.org/pkg/xparse
\usepackage{etoolbox}% http://ctan.org/pkg/etoolbox
\newcounter{listtotal}\newcounter{listcntr}%
\NewDocumentCommand{\entries}{o}{%
	\setcounter{listtotal}{0}\setcounter{listcntr}{-1}%
	\renewcommand*{\do}[1]{\stepcounter{listtotal}}%
	\expandafter\docsvlist\expandafter{\entriesarray}%
	\IfNoValueTF{#1}
	{\entriesarray}% \names
	{% \names[<index>]
		\renewcommand*{\do}[1]{\stepcounter{listcntr}\ifnum\value{listcntr}=#1\relax##1\fi}%
		\expandafter\docsvlist\expandafter{\entriesarray}}%
}
\newcommand{\entriesarray}{}

\lhead{}
\rhead{CSC363H5 Winter 2023} 
\chead{\textbf{Tutorial 3: Recognizable and Decidable Sets}}
\date{}
\begin{document}
		
\subsection*{Things to remember}


\begin{itemize}
	\item Let $\Sigma$ be the input alphabet. We let $\Sigma^*$ denote the set of finite strings containing characters from $\Sigma$.  For example, suppose $\Sigma = \{\mathrm{a}, \mathrm{c}, \mathrm{d}, \mathrm{e}, \mathrm{i}, \mathrm{m}\}$. Then  $\Sigma^*$ contains the strings $\mathrm{mmmmmddddddd}$, $\mathrm{mid}$, $\mathrm{academia}$, $\mathrm{macademia}$, and the empty string $\epsilon$, but does not contain the strings $\mathrm{bad}$ or $\mathrm{13d}$.
	\item A subset of $\Sigma^*$ is called a \textbf{language}. With $\Sigma = \{\mathrm{a}, \mathrm{c}, \mathrm{d}, \mathrm{e}, \mathrm{i}, \mathrm{m}\}$ again, examples of languages include the empty language $\emptyset$, the universal language  $\Sigma^*$, the singleton language $\{\mathrm{a}\}$, and so on.
	\item A language $L$ is \textbf{decidable} if there is a Turing machine $M$ such that for any input $s \in \Sigma^*$:
	\begin{itemize}
		\item If $s \in L$, then $M$ accepts $s$.
		\item If $s \notin L$, then $M$ \underline{rejects} $s$.
	\end{itemize}
	\item A language $L$ is \textbf{recognizable} if there is a Turing machine $M$ such that for any input $s \in \Sigma^*$:
	\begin{itemize}
		\item If $s \in L$, then $M$ accepts $s$.
		\item If $s \notin L$, then $M$ \underline{rejects or loops} on $s$.
	\end{itemize}
	\item A function $f: \Sigma^* \to \Sigma^*$ is \textbf{computable} if there is a Turing machine $M$ such that for any input $s \in \Sigma^*$:
	\begin{itemize}
		\item $M$ halts (either accepts or rejects).
		\item $M$ outputs $f(s)$.
	\end{itemize}
\end{itemize}

\subsection*{Dealing with an identity crisis}
Recognizable/decidable languages are central notions to computability theory. They also go by many names!
\begin{itemize}
	\item Recognizable languages are sometimes called \textit{recursively enumerable languages}.
	\item Decidable languages are sometimes called \textit{recursive} or \textit{computable} languages.
\end{itemize}
\subsection*{Exercises}

\textbf{Exercise -1.} Let $\Sigma = \{0, 1\}$ be the input alphabet. Show that the following languages are decidable. Do not formally construct the Turing machines; simply describe how a Turing machine (or equivalently, a computer program) would decide if an input is in the language.

\begin{itemize}
	\item $\emptyset$.
	\item $\{ww: w \in \Sigma^*\}$.
	\item $\{w \in \Sigma^*: \text{$w$ is a prime number in binary}\}$.
	\item $\{w \in \Sigma^*: \text{There is an integer solution $x$ to $wx^6 - 5wx^5 + 4000x^4 - 2wx - 60 = 0$}\}$.
	\vspace{-3mm}
\end{itemize}

\noindent \hrulefill

\textbf{Exercise 0.} Show that a language $L$ is decidable if and only if its \textit{characteristic function} $\chi_L$ is computable, where
\begin{align*}
	\chi_L(s) &=
	\begin{cases}
		1, & s\in L,\\
		0, & s \notin L.
	\end{cases}
\end{align*}

\noindent \hrulefill

\textbf{Exercise 1.} Let $L_1, L_2$ be languages.
\begin{enumerate}[label=(\alph*)]
 
\item If $L_1$ and $L_2$ are decidable, prove that the following sets are decidable, by giving a high-level description/pseudocode of an appropriate Turing machine.
\begin{multicols}{3}
\begin{enumerate}[label=(\roman*)]
\item $L_1 \cup L_2$.
\item $L_1 \cap L_2$.
\item $L_1^\text{c}$.
\end{enumerate}
\end{multicols}

 \item If $L_1$ and $L_2$ are recognizable, prove that the following sets are recognizable.
\begin{enumerate}[label=(\roman*)]
\begin{multicols}{3}
\item $L_1 \cup L_2$.
\item $L_1 \cap L_2$.
\item $L_1L_2 = \{st: s \in L_1,\; t \in L_2\}$.
\end{multicols}
\end{enumerate}
\item Is the complement of a recognizable language always recognizable? If not, when is this true?
\end{enumerate}

\noindent \hrulefill

\textbf{Exercise 2.} Show that every infinite recognizable language has an infinite decidable subset.
\end{document}
