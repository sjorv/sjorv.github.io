\documentclass[10pt]{article}
%%%%%%%%%%%%%%%%%%%%%%%%%%%%%%%%%%%%%%%%%
% Specification file for
% Custom Assignment Template (For Students) by Paul Zhang
%
% Based on:
% Cleese Assignment (For Students) Version 2.0 (27/5/2018)
%
% This template originates from http://www.LaTeXTemplates.com
% Original Author: Vel (vel@LaTeXTemplates.com)
%
% License:
% CC BY-NC-SA 3.0 (http://creativecommons.org/licenses/by-nc-sa/3.0/)
%
%%%%%%%%%%%%%%%%%%%%%%%%%%%%%%%%%%%%%%%%%

%----------------------------------------------------------------------------------------
%	PACKAGES AND OTHER DOCUMENT CONFIGURATIONS
%----------------------------------------------------------------------------------------

\usepackage{
amsfonts,     % Extra fonts
amsmath,      % Lots of useful mathematical symbols
amssymb,      % More useful mathematical symbols
amsthm,       % Theorem and proof environments
cite,         % Citations
enumitem,     % Change enumerate counter style
fancyhdr, 	  % Fancy header
float,		  % Figure placement
geometry,     % Required for adjusting page dimensions and margins
graphicx,     % Required to insert images
hyperref,     % Hyperlinks
ifthen,       % Required for ifthenelse statements
import,       % So that tex files can input from relative paths
listings,     % Required for insertion of code
mathtools,    % Extra math symbols
multicol,     % Easy setup for local multi-column use
pgfplots,     % For drawing graphs of functions
tikz,         % Nice diagrams
xcolor,       % Used for changing text color
}

\pgfplotsset{compat=1.17}
\hypersetup{
  colorlinks   = true,    % Colours links instead of ugly boxes
  urlcolor     = blue,    % Colour for external hyperlinks
  linkcolor    = blue,    % Colour of internal links
  citecolor    = red      % Colour of citations
}

\usepackage[most]{tcolorbox} % Required for boxes that split across pages
\usepackage[justification=centering, margin=1.3cm]{caption} % Center long captions

\setlength{\parindent}{0pt}

%----------------------------------------------------------------------------------------
%	LISTINGS
%----------------------------------------------------------------------------------------

\definecolor{codegreen}{rgb}{0,0.6,0}
\definecolor{codegray}{rgb}{0.5,0.5,0.5}
\definecolor{codepurple}{rgb}{0.58,0,0.82}
\definecolor{backcolour}{rgb}{0.95,0.95,0.92}

\lstdefinestyle{mystyle}{
	backgroundcolor=\color{backcolour},
	commentstyle=\color{codegreen},
	keywordstyle=\color{magenta},
	numberstyle=\tiny\color{codegray},
	stringstyle=\color{codepurple},
	basicstyle=\ttfamily,
	breakatwhitespace=false,
	breaklines=true,
	captionpos=b,
	keepspaces=true,
	numbers=left,
	numbersep=5pt,
	showspaces=false,
	showstringspaces=false,
	showtabs=false,
	tabsize=2
}

\lstdefinestyle{mystyletiny}{
	backgroundcolor=\color{backcolour},
	commentstyle=\color{codegreen},
	keywordstyle=\color{magenta},
	numberstyle=\tiny\color{codegray},
	stringstyle=\color{codepurple},
	basicstyle=\tiny\ttfamily,
	breakatwhitespace=false,
	breaklines=true,
	captionpos=b,
	keepspaces=true,
	numbers=left,
	numbersep=5pt,
	showspaces=false,
	showstringspaces=false,
	showtabs=false,
	tabsize=2
}

\lstset{style=mystyle}

%----------------------------------------------------------------------------------------
%	MARGINS
%----------------------------------------------------------------------------------------

\usepackage{geometry} % Required for adjusting page dimensions and margins

\geometry{
	paper=a4paper, % Change to letterpaper for US letter
	top=1.7cm, % Top margin
	bottom=1.7cm, % Bottom margin
	left=2.5cm, % Left margin
	right=2.5cm, % Right margin
	headheight=14pt, % Header height
	footskip=0.5cm, % Space from the bottom margin to the baseline of the footer
	headsep=0.5cm, % Space from the top margin to the baseline of the header
	%showframe, % Uncomment to show how the type block is set on the page
}

%----------------------------------------------------------------------------------------
%	FONT
%----------------------------------------------------------------------------------------

\usepackage[utf8]{inputenc} % Required for inputting international characters
\usepackage[T1]{fontenc} % Output font encoding for international characters

\usepackage{tgpagella} % Use the TEX Gyre Pagella font

%----------------------------------------------------------------------------------------
%	HEADERS AND FOOTERS
%----------------------------------------------------------------------------------------

\usepackage{fancyhdr} % Required for customising headers and footers

\pagestyle{fancy} % Enable custom headers and footers

\lhead{} % Left header
\chead{} % Centre header
\rhead{}

\lfoot{} % Left footer
% \cfoot{} % Centre footer, shows page numbers
\rfoot{} % Right footer

\renewcommand\headrulewidth{0pt} % Remove header rule

%----------------------------------------------------------------------------------------
%	MODIFY SECTION STYLES
%----------------------------------------------------------------------------------------

\usepackage{titlesec} % Required for modifying sections

\pagenumbering{arabic}

%------------------------------------------------
% Section

\titleformat
{\section} % Section type being modified
[block] % Shape type, can be: hang, block, display, runin, leftmargin, rightmargin, drop, wrap, frame
{\LARGE\bfseries} % Format of the whole section
{\assignmentQuestionName~\thesection} % Format of the section label
{6pt} % Space between the title and label
{} % Code before the label

\titlespacing{\section}{0pt}{0.5\baselineskip}{0.5\baselineskip} % Spacing around section titles, the order is: left, before and after

%------------------------------------------------
% Subsection

\titleformat
{\subsection} % Section type being modified
[block] % Shape type, can be: hang, block, display, runin, leftmargin, rightmargin, drop, wrap, frame
{\large\bfseries} % Format of the whole section
{(\alph{subsection})} % Format of the section label
{4pt} % Space between the title and label
{} % Code before the label

\titlespacing{\subsection}{0pt}{0.5\baselineskip}{0.5\baselineskip} % Spacing around section titles, the order is: left, before and after

\renewcommand\thesubsection{(\alph{subsection})}

%------------------------------------------------
% Theorem, Lemma, etc

\newenvironment{question}[1][]{
    {\noindent \textbf{\ifthenelse{\equal{#1}{}}{Question. }{Question [#1]. }}}}
    {}

\newenvironment{definition}[1][]{
    {\noindent \textbf{\ifthenelse{\equal{#1}{}}{Definition. }{Definition [#1]. }}}}
    {}

\newenvironment{solution}{
    ~\\ \noindent \textbf{My solution.} \par}
    {}

\newenvironment{theorem}[1][]{
    {\noindent \textbf{\ifthenelse{\equal{#1}{}}{Theorem. }{Theorem [#1]. }}}}
    {}

\newenvironment{lemma}[1][]{
    {\noindent \textbf{\ifthenelse{\equal{#1}{}}{Lemma. }{Lemma [#1]. }}}}
    {}

\newenvironment{proposition}[1][]{
    {\noindent \textbf{\ifthenelse{\equal{#1}{}}{Proposition. }{Proposition [#1]. }}}}
    {}

\renewenvironment{comment}{
    {\noindent \textbf{Comment. }}}
    {}


%------------------------------------------------
% Math Commands

\newcommand{\N}{\mathbb N}        % Naturals
\newcommand{\Z}{\mathbb Z}        % Integers
\newcommand{\Q}{\mathbb Q}        % Rationals
\newcommand{\R}{\mathbb R}        % Reals
\newcommand{\C}{\mathbb C}	      % Complex
\newcommand{\mb}{\mathbf}
\newcommand{\mc}{\mathcal}
\newcommand{\mf}{\mathfrak}
\newcommand{\inver}{^{-1}}
\newcommand{\floor}[1]{\lfloor #1 \rfloor}
\newcommand{\ceil}[1]{\lceil #1 \rceil}

%------------------------------------------------
% Tikz settings
\usetikzlibrary{shapes.geometric}
\usetikzlibrary{arrows.meta}
\usetikzlibrary{automata, calc, positioning}

\usepackage{xparse}% http://ctan.org/pkg/xparse
\usepackage{etoolbox}% http://ctan.org/pkg/etoolbox
\newcounter{listtotal}\newcounter{listcntr}%
\NewDocumentCommand{\entries}{o}{%
	\setcounter{listtotal}{0}\setcounter{listcntr}{-1}%
	\renewcommand*{\do}[1]{\stepcounter{listtotal}}%
	\expandafter\docsvlist\expandafter{\entriesarray}%
	\IfNoValueTF{#1}
	{\entriesarray}% \names
	{% \names[<index>]
		\renewcommand*{\do}[1]{\stepcounter{listcntr}\ifnum\value{listcntr}=#1\relax##1\fi}%
		\expandafter\docsvlist\expandafter{\entriesarray}}%
}
\newcommand{\entriesarray}{}

\lhead{}
\rhead{CSC363H5 Winter 2023} 
\chead{\textbf{Tutorial 9: Boolean Satisfiability Problem}}
\date{}
\begin{document}
		
In this worksheet, we outline a proof that SAT is NP-Complete (known as the \textbf{Cook-Levin Theorem}) (from \href{https://en.wikipedia.org/wiki/Cook%E2%80%93Levin_theorem#Proof}{\textcolor{blue}{Wikipedia}}). This proof has two parts.
\begin{enumerate}
	\item $\mathrm{SAT} \in \mathrm{NP}$.
	\item SAT is \textit{NP-hard}: Any language $L \in \mathrm{NP}$ will satisfy $L \leq_p \mathrm{SAT}$.
\end{enumerate}


\noindent \textbf{Exercise 1.} Show that $\mathrm{SAT} \in \mathrm{NP}$. That is, build a poly-time nondeterministic Turing machine that decides SAT.

\subsection*{SAT is NP-hard}

Let $L \in \mathrm{NP}$. We will show that $L \leq_p \mathrm{SAT}$ by constructing a poly-time computable function $f$ such that $x \in L \Leftrightarrow f(x) \in \mathrm{SAT}$.

Since $L \in \mathrm{NP}$, there is a poly-time nondeterministic Turing machine $M = (Q, \Sigma, \Gamma, \delta, q_0, q_\text{acc}, q_\text{rej})$ that decides $L$. Since $M$ is poly-time, we may assume $M$ halts in $\leq p(n)$ steps on any input of size $n$ (where $p(n)$ is some polynomial). We will make use of the following lemma:

\vspace{2mm}

\begin{lemma}
	Suppose $M(x)$ finishes executing in $p(n)$ steps. Then during execution, only cells $-p(n)$ to $p(n)$ could have been accessed during execution; cells $-\infty$ to $-p(n) - 1$, and cells $p(n) + 1$ to $\infty$ are inaccessible to the read/write head.
\end{lemma}

\begin{center}
	\begin{tikzpicture}[scale=0.5]
		% tape
		\renewcommand{\entriesarray}{$-5$, $-4$, $-3$, $-2$, $-1$, $0$, $1$, $2$, $3$, $4$, $5$}
		\pgfmathsetmacro{\NumCells}{11}
		\pgfmathsetmacro{\NumCellsMinusOne}{\NumCells - 1}
		\foreach \x in {0, ...,\NumCellsMinusOne} {
			\draw (\x, 0) rectangle (\x + 1, 1);
			\node at (\x + 0.5, 1.2) {\footnotesize\texttt{\entries[\x]}};
		}
		\draw (-0.5, 0) -- (0, 0); \draw (-0.5, 1) -- (0, 1); \node[left] at (-0.5, 0.5) {$\ldots$}; 
		\draw (\NumCells, 0) -- (\NumCells + 0.5, 0); \draw (\NumCells, 1) -- (\NumCells + 0.5, 1); \node[right] at (\NumCells + 0.5, 0.5) {$\ldots$}; 
		
		% head
		\pgfmathsetmacro{\HeadPosition}{5}
		\draw (\HeadPosition - 0.5, -2.5) rectangle (\HeadPosition + 1.5, -1.5);
		\node at (\HeadPosition + 0.5, -2) {Head};
		\draw[->] (\HeadPosition + 0.5, -1.5) -- (\HeadPosition + 0.5, 0);
	\end{tikzpicture}
\end{center}

\noindent \textbf{Exercise 2.} Briefly justify the Lemma. \textit{Hint: How long does it take to move the read/write head?}
\vspace{2mm}

The Lemma guarantees that if $M$ is given an input $x$ of size $n$, then throughout $M(x)$'s execution, only cells $-p(n)$ to $p(n)$ can be overwritten.


Now, given an input $x$, we will define the following collections of variables:

\begin{table}[ht]
	\centering
	\begin{tabular}{c|c|c}
		Variables & Range & Intended Interpretation\\
		\hline 
		$T_{i, j, k}$ & $-p(n) \leq i \leq p(n)$, $j \in \Sigma$, $0 \leq k \leq p(n)$ & Cell $i$ has symbol $j$ at step $k$ of execution.\\
		\hline
		$H_{i, k}$ & $-p(n) \leq i \leq p(n)$, $0 \leq k \leq p(n)$ & The read/write head is at cell $i$ at step $k$ of execution.\\
		\hline
		$Q_{q, k}$ & $q \in Q$, $0 \leq k \leq p(n)$ & The NTM is at state $q$ at step $k$ of execution.
	\end{tabular}
	\caption{Variables and their intended interpretations}
	\label{table:1}
\end{table}

\noindent \textbf{Exercise 3.} Suppose $M$ is the \textit{deterministic} Turing machine below (with $\Sigma = \{a, b\}$). You may assume that $M(x)$ halts in $p(n) = n + 1$ steps or less.


\begin{center}
	\begin{tikzpicture}[shorten >=1pt,node distance=2.5cm,on grid,auto]
		\node[state] (q0) [] {$q_0$};
		\node[state] (q1) [right of=q0] {$q_1$};
		\node[state, accepting] (qacc) [right of=q1] {$q_\text{acc}$};
		\node[state] (qrej) [left of=q0] {$q_\text{rej}$};
		
		\path[->] (q0) edge node[above] {$(a, b, R)$} (q1);
		\path[->] (q0) edge node[above] {$(\square, \square, R)$} (qrej);
		\path[->] (q0) edge [loop below] node[below] {$(b, a, R)$} (q0);
		\path[->] (q1) edge node[above] {$(\square, \square, R)$} (qacc);
		\path[->] (q1) edge [bend right=70] node[below] {$(b, b, R)$} (qrej);
		\path[->] (q1) edge [loop above] node[above] {$(a, a, R)$} (q1);
	\end{tikzpicture}
\end{center}
\noindent Execute $M$ on the input $x = bba$ (so $M$ halts in $p(n) = 3 + 1 = 4$ steps), and label the variables below with their intended interpretations (consulting Table \ref{table:1}).
\begin{center}
	\begin{tikzpicture}[scale=0.5]
		% tape
		\renewcommand{\entriesarray}{$-5$, $-4$, $-3$, $-2$, $-1$, $0$, $1$, $2$, $3$, $4$, $5$}
		\pgfmathsetmacro{\NumCells}{11}
		\pgfmathsetmacro{\NumCellsMinusOne}{\NumCells - 1}
		\foreach \x in {0, ...,\NumCellsMinusOne} {
			\draw (\x, 0) rectangle (\x + 1, 1);
			\node at (\x + 0.5, 1.2) {\footnotesize\texttt{\entries[\x]}};
		}
		\node at (5.5, 0.5) {$b$};
		\node at (6.5, 0.5) {$b$};
		\node at (7.5, 0.5) {$a$};
		\draw (-0.5, 0) -- (0, 0); \draw (-0.5, 1) -- (0, 1); \node[left] at (-0.5, 0.5) {$\ldots$}; 
		\draw (\NumCells, 0) -- (\NumCells + 0.5, 0); \draw (\NumCells, 1) -- (\NumCells + 0.5, 1); \node[right] at (\NumCells + 0.5, 0.5) {$\ldots$}; 
		
		% head
		\pgfmathsetmacro{\HeadPosition}{5}
		\draw (\HeadPosition - 0.5, -2.5) rectangle (\HeadPosition + 1.5, -1.5);
		\node at (\HeadPosition + 0.5, -2) {Head};
		\draw[->] (\HeadPosition + 0.5, -1.5) -- (\HeadPosition + 0.5, 0);
	\end{tikzpicture}
\end{center}

\begin{minipage}{0.22\textwidth}
	\begin{tabular}{c|c}
		Variable & Value\\
		\hline 
		$T_{0, a, 0}$ & F\\
		$T_{1, b, 0}$ & T\\
		$T_{2, b, 0}$ & F\\
		$T_{0, a, 1}$ & \\
		$T_{1, a, 1}$ & \\
		$T_{2, a, 2}$ & \\
	\end{tabular}
\end{minipage}
\begin{minipage}{0.22\textwidth}
	\begin{tabular}{c|c}
		Variable & Value\\
		\hline 
		$T_{3, \square, 2}$ & \\
		$T_{-2, \square, 2}$ & \\
		$T_{2, b, 2}$ & \\
		$T_{0, a, 3}$ & \\
		$T_{1, a, 4}$ & \\
		$T_{2, a, 4}$ & \\
	\end{tabular}
\end{minipage}
\begin{minipage}{0.22\textwidth}
	\begin{tabular}{c|c}
		Variable & Value\\
		\hline 
		$H_{0, 0}$ & T\\
		$H_{1, 0}$ & F\\
		$H_{1, 1}$ & T\\
		$H_{2, 1}$ & \\
		$H_{2, 2}$ & \\
		$H_{3, 3}$ & \\
	\end{tabular}
\end{minipage}
\begin{minipage}{0.22\textwidth}
	\begin{tabular}{c|c}
		Variable & Value\\
		\hline 
		$Q_{q_0, 0}$ & T\\
		$Q_{q_1, 0}$ & F\\
		$Q_{q_1, 1}$ & F\\
		$Q_{q_1, 2}$ & \\
		$Q_{q_1, 3}$ & \\
		$Q_{q_\text{acc}, 4}$ & \\
	\end{tabular}
\end{minipage}

\newpage

In Table \ref{table:1}, there are:
\begin{itemize}
	\item $(2p(n) + 1) \cdot |\Sigma| \cdot (p(n) + 1) = \mathcal O(p(n)^2)$ variables of the form $T_{i, j, k}$.
	\item $(2p(n) + 1) \cdot (p(n) + 1) = \mathcal O(p(n)^2)$ variables of the form $H_{i, k}$.
	\item $|Q| \cdot (p(n) + 1) = \mathcal O(p(n))$ variables of the form $Q_{q, k}$.
\end{itemize}
In total, we have created $\mathcal O(p(n)^2)$ variables given an input $x$ of size $n$.


Recall that we want to create a poly-time computable $f$ so that $x \in L \Leftrightarrow f(x) \in \mathrm{SAT}$. In other words, given an $x$, we want to create a boolean formula $f(x)$ in polynomial time, so that $f(x)$ is satisfiable iff $x \in L$. Here's how we create this boolean formula $f(x)$:
\begin{itemize}
	\item The variables of this boolean formula are the $T_{i, j, k}$'s, the $H_{i, k}$'s, and the $Q_{q, k}$'s, as defined in Table \ref{table:1}. There are $\mathcal O(p(n)^2)$ variables.
	\item The formula is the \textit{conjunction} ($\land$) of all of the following boolean subformulae (with $-p(n) \leq i \leq p(n)$, $0 \leq k \leq p(n)$):
	
	\begin{table}[ht]
		\centering
		\begin{tabular}{c|c|c|c}
			Formulae & Range & Interpretation & How many formulae?\\
			\hline
			$T_{i, j, 0}$ & \begin{tabular}{c}
				$j \in \Sigma$\\
				cell $i$ initially\\
				contains symbol $j$\\
			\end{tabular} & Initial contents of the tape & $\mathcal O(p(n))$\\
			\hline
			$Q_{q_0, 0}$ &  & TM starts in state $q_0$ & $1$\\
			\hline
			$H_{0, 0}$ &  & R/W head starts at cell 0 & $1$\\
			\hline
			$\neg T_{i, j, k} \lor \neg T_{i, j', k}$ & $j, j' \in \Sigma$ with $j \neq j'$ & At most 1 symbol per cell & $\mathcal O(p(n)^2)$\\
			\hline
			$\displaystyle \bigvee_{j \in \Sigma} T_{i, j, k}$ & & At least 1 symbol per cell & $\mathcal O(p(n)^2)$\\
			\hline
			$T_{i, j, k} \land T_{i, j', k+1} \rightarrow H_{i, k}$ &
				$j, j' \in \Sigma$ with $j \neq j'$ & 
				\begin{tabular}{c}
				To change cell $i$,\\
				head must be at cell $i$
				\end{tabular} & $\mathcal O(p(n)^2)$\\
			\hline
			$\neg Q_{q, k} \lor \neg Q_{q', k}$ & $q, q' \in Q$ with $q \neq q'$ & Only one state at a time & $\mathcal O(p(n))$\\	
			\hline
			$\neg H_{i, k} \lor \neg H_{i', k}$ & $i \neq i'$ & 
			\begin{tabular}{c}
				Only one head \\
				position at a time
			\end{tabular} & $\mathcal O(p(n)^3)$\\	
			\hline
			\begin{tabular}{c}
				$(H_{i, k} \land Q_{q, k} \land T_{i, j, k}) \rightarrow$\\
				{\footnotesize$\displaystyle \bigvee_{(q, j), (q', j', d) \in \delta}$
				$\left(\begin{array}{c}
					H_{i + d, k+1}\\
					\land Q_{q', k+1}\\
					\land T_{i, j', k + 1}
				\end{array}\right)$}
			\end{tabular} & \begin{tabular}{c}
			$j \in \Sigma$\\
			$q \in Q$\\
			$k \neq p(n)$
			\end{tabular} & \begin{tabular}{c}
			Non-deterministic transition\\
			function $\delta$ is obeyed
			\end{tabular} & $\mathcal O(p(n)^2)$\\
			\hline
			$\displaystyle \bigvee_{0 \leq k \leq p(n)} Q_{q_\text{acc}, k}$ & & \begin{tabular}{c}
				Accepting state $q_\text{acc}$\\ 
				reached within $p(n)$ steps
			\end{tabular} & $1$
		\end{tabular}
	\caption{$f(x)$ is the conjunction ($\land$) of all of the following subformulae.}
	\label{table:2}
	\end{table}
\end{itemize}
Notice that this huge boolean formula $f(x)$ is satisfiable iff $x \in L$:
\begin{itemize}
	\item If $f(x)$ is satisfiable, then there is some execution path in $M(x)$ that ends in the accepting state $q_\text{acc}$. It follows from the definition (since $M$ is a nondeterministic decider for $L$) that $x \in L$.
	\item If $x \in L$, then there is some execution path in $M(x)$ leading to acceptance. We may then assign the $T_{i, j, k}$'s, the $H_{i, k}$'s, and the $Q_{q, k}$'s according to their intended interpretation to satisfy $f(x)$.
\end{itemize}
$f(x)$ takes $\mathcal O(p(n)^3)$ time to produce, and since $p(n)$ is polynomial, so is $p(n)^3$. Thus $f$ is indeed poly-time computable. This shows $L \leq_p \mathrm{SAT}$. $\square$

\vspace{2mm}


\noindent\textbf{Exercise 4.} Referring back to the Turing machine $M$ in Exercise 3, $M$ accepts the input $x = bba$. 

\begin{enumerate}[label=(\alph*)]
	\item List the variables in $f(x)$ (consulting Table \ref{table:1}).
	\item Write down the boolean formula $f(x)$ (consulting table \ref{table:2}).
\end{enumerate} 


\end{document}
