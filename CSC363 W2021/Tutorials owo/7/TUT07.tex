\documentclass{beamer}

\usetheme{simple}

\usepackage{scalerel,xparse}
\usepackage{lmodern}
\usepackage[scale=2]{ccicons}
\usepackage{ulem}
\usepackage{tikz}
\usetikzlibrary{positioning,calc,automata}
\usepackage{algorithm}
\usepackage{algorithmic}
\usepackage{caption}
\usepackage{listings}
\usepackage{xcolor}

% Watermark background (simple theme)
\setlength{\parindent}{0cm}
\setwatermark{\includegraphics[height=8cm]{img/chungus_glitch.png}}


\title{CSC363 Tutorial 7}
\subtitle{\includegraphics[width=4cm]{img/qr.png}}
\date{\today}
\author{Paul ``sushi{\textunderscore}enjoyer'' Zhang}
\institute{University of Humongus Chungus Amogus}

\NewDocumentCommand\emojisushi{}{
    \includegraphics{img/1f363.png}
}
\NewDocumentCommand\emojiflushed{}{
    \includegraphics[scale=0.05]{img/1f633.png}
}  
\NewDocumentCommand\emojimoyai{}{
    \includegraphics{img/1f5ff.png}
}   

\begin{document}

\maketitle

\begin{frame}{Learning objectives this tutorial}
By the end of this tutorial, you should...
\begin{itemize}
\item Have \url{turingmachinesimulator.com} bookmarked in Microsoft Edge (or whichever browser you use).
\item Feel great, because you will probably never have to touch upon any computability theory again.\footnote{Oh wait! You have an assignment due on computability theory. ;-;
If you want \sout{unauthorized aids} help, stay for my office hours! :D}
\item Understand $p$-reducibility and have a practical example in the back of your mind.
\item Fall in love with polynomials. They are so nice!
\end{itemize}
(no readings this week cuz we didn't really go over anything new)
\end{frame}

\begin{frame}{More Turing Machines. \emojimoyai}

\textbf{Task:} Answer the following:
\begin{quote}
do you like optimwizing cowde? uwu
\end{quote}

\begin{figure}[h]
\centering
\includegraphics[width=5cm]{img/gru.jpg}
\end{figure}

\pause

\textbf{Answer:} Of course you do.


\end{frame}


\begin{frame}{Turing machines are back! :D}

\textbf{Task:} Decrypt what this Turing machine is doing (i.e. find the language that this Turing machine accepts). Our alphabet is $\{0, 1\}$.

\begin{center}
\begin{tabular}{c|c|c|c}
State & 0 & 1 & $\square$\\
\hline
$q_0$ & ($q_1$, $\square$, R) & reject & accept\\
\hline
$q_1$ & ($q_1$, 0, R) & ($q_1$, 1, R) & ($q_2$, $\square$, L)\\
\hline
$q_2$ & reject & ($q_3$, $\square$, L) & reject\\
\hline
$q_3$ & ($q_3$, 0, L) & ($q_3$, 1, L) & ($q_0$, $\square$, R)\\
\end{tabular}
\end{center}

Hint: 

\begin{figure}[h]
\centering
\includegraphics[width=6cm]{img/dejavu.jpg}
\end{figure}

\end{frame}

\begin{frame}{Turing machines are back! :D}

\textbf{Task:} Decrypt what this Turing machine is doing (i.e. find the language that this Turing machine accepts). Our alphabet is $\{0, 1\}$.

\begin{center}
\begin{tabular}{c|c|c|c}
State & 0 & 1 & $\square$\\
\hline
$q_0$ & ($q_1$, $\square$, R) & reject & accept\\
\hline
$q_1$ & ($q_1$, 0, R) & ($q_1$, 1, R) & ($q_2$, $\square$, L)\\
\hline
$q_2$ & reject & ($q_3$, $\square$, L) & reject\\
\hline
$q_3$ & ($q_3$, 0, L) & ($q_3$, 1, L) & ($q_0$, $\square$, R)\\
\end{tabular}
\end{center}

Answer: $\{0^n 1^n: n \in \mathbb N\}$. (this is the same turing machine as Q1 Assignment 1)

\end{frame}

\begin{frame}{Turing machines are back! :D}

\begin{center}
\begin{tabular}{c|c|c|c}
State & 0 & 1 & $\square$\\
\hline
$q_0$ & ($q_1$, $\square$, R) & reject & accept\\
\hline
$q_1$ & ($q_1$, 0, R) & ($q_1$, 1, R) & ($q_2$, $\square$, L)\\
\hline
$q_2$ & reject & ($q_3$, $\square$, L) & reject\\
\hline
$q_3$ & ($q_3$, 0, L) & ($q_3$, 1, L) & ($q_0$, $\square$, R)\\
\end{tabular}
\end{center}

\textbf{Task:} Open Microsoft Edge (or whatever your favourite browser is :D), go to \url{http://turingmachinesimulator.com/shared/bkuepwxgdh}. Bookmark the website. Put on your favourite music. Read the code. Click ``Compile''. Try a few different inputs and see how the Turing machine runs. Convince yourself that this Turing machine indeed accepts exactly $\{0^n 1^n: n \in \mathbb N\}$. Go grab a snack in the meantime.

\end{frame}

\begin{frame}{Turing machines are back! :D}

\begin{center}
\begin{tabular}{c|c|c|c}
State & 0 & 1 & $\square$\\
\hline
$q_0$ & ($q_1$, $\square$, R) & reject & accept\\
\hline
$q_1$ & ($q_1$, 0, R) & ($q_1$, 1, R) & ($q_2$, $\square$, L)\\
\hline
$q_2$ & reject & ($q_3$, $\square$, L) & reject\\
\hline
$q_3$ & ($q_3$, 0, L) & ($q_3$, 1, L) & ($q_0$, $\square$, R)\\
\end{tabular}
\end{center}

\textbf{Task:} Open Microsoft Edge (or whatever your favourite browser is :D), go to \url{http://turingmachinesimulator.com/shared/bkuepwxgdh}. Bookmark the website. Put on your favourite music. Read the code. Click ``Compile''. Try a few different inputs and see how the Turing machine runs. Convince yourself that this Turing machine indeed accepts exactly $\{0^n 1^n: n \in \mathbb N\}$. Also convince yourself that this is an $O(n^2)$ Turing machine: given an input string of length $n$, this Turing machine takes at most $O(n^2)$ steps to halt (no matter if it accepts or rejects).

\end{frame}

\begin{frame}{Turing machines are back! :D}

\begin{center}
\begin{tabular}{c|c|c|c}
State & 0 & 1 & $\square$\\
\hline
$q_0$ & ($q_1$, $\square$, R) & reject & accept\\
\hline
$q_1$ & ($q_1$, 0, R) & ($q_1$, 1, R) & ($q_2$, $\square$, L)\\
\hline
$q_2$ & reject & ($q_3$, $\square$, L) & reject\\
\hline
$q_3$ & ($q_3$, 0, L) & ($q_3$, 1, L) & ($q_0$, $\square$, R)\\
\end{tabular}
\end{center}

\textbf{Task:} Open Microsoft Edge (or whatever your favourite browser is :D), go to \url{http://turingmachinesimulator.com/shared/bkuepwxgdh}. Bookmark the website. Put on your favourite music. Read the code. Click ``Compile''. Try a few different inputs and see how the Turing machine runs. Convince yourself that this Turing machine indeed accepts exactly $\{0^n 1^n: n \in \mathbb N\}$. Also convince yourself that this is an $O(n^2)$ Turing machine: given an input string of length $n$, this Turing machine takes at most $O(n^2)$ steps to halt (no matter if it accepts or rejects).

\end{frame}

\begin{frame}{Turing machines are back! :D}
But we can do better, in fact! There is an $O(n \log n)$ way to decide whether a given input is in ${0^n 1^n: n \in \mathbb N}$. Here is the high-level idea:

On input $w$, we do the following:
\begin{enumerate}
\item Scan across the tape and reject if a 0 is found to the right of a 1.
\item Repeat the following as long as there is both a 0 and 1 on the tape:
\begin{enumerate}
\item Scan across the tape, checking if the total number of 0s and 1s remaining is odd. If it is odd, \textit{reject}.
\item Scan again across the tape, crossing off every other 0 starting with the first 0, and then crossing off every other 1 starting with the first 1.
\end{enumerate}
\item If the tape doesn't have any 0s or 1s, \textit{accept}. Else, \textit{reject}.
\end{enumerate}

\textbf{Task: } Try this procedure (by hand) on input $0000011111$ and on input $000001111$.
\end{frame}

\begin{frame}{Turing machines are back! :D}
\begin{enumerate}
\item Scan across the tape and reject if a 0 is found to the right of a 1.
\item Repeat the following as long as there is both a 0 and 1 on the tape:
\begin{enumerate}
\item Scan across the tape, checking if the total number of 0s and 1s remaining is odd. If it is odd, \textit{reject}.
\item Scan again across the tape, crossing off every other 0 starting with the first 0, and then crossing off every other 1 starting with the first 1.
\end{enumerate}
\item If the tape doesn't have any 0s or 1s, \textit{accept}. Else, \textit{reject}.
\end{enumerate}
The above procedure actually takes $O(n \log n)$ time for the Turing machine, because we cross off at least half of all the symbols on each iteration of 2, so $2$ runs at most $\log n$ times (of course i'm being informal here).

But either way, it's more efficient now! yay\footnote{yea optimizing turing machine code is definitely something i would do as a career.}

\vspace{2mm}

You may try it out at \url{http://turingmachinesimulator.com/shared/prsswhkkyb}.

\end{frame}

\begin{frame}{i dunno... break time before we get into $P$-reducibility?}
hmm... i'm not sure if youse prefer watching Turing machines execute, over playing with computable and c.e. sets!\footnote{i don't think you want to ever see the symbol $\varphi$ again.}

\begin{block}{Trivia time!}
What ingredients are in this ``drink''?\footnote{i dunno if this classifies as a drink...}

\pause

\textbf{Answer:} i dunno, i haven't yet made the drink when making these slides.
\end{block}

\begin{figure}[h]
\centering
\includegraphics[width=6cm]{img/helo_fish_sushi.jpg}
\end{figure}

 
\end{frame}

\begin{frame}{some preface as we're transitioning into a different part of the course, probably} 

Remember when we were talking about computability of \textit{sets of natural numbers} $A \subseteq \mathbb N$?

\vspace{2mm}

Now we are talking about computability of \textit{languages} $L \subseteq \Sigma^*$, which are sets of strings. 

\vspace{2mm}

But really, they're the same thing! Each string has a G*del number. When we talk about a language $L \subseteq \Sigma^*$, we can instead be talking about
$$A = \{e \in \mathbb N: \text{$e$ is the G*dot number for some string in $L$}\}$$
and vice versa.

\end{frame}

\begin{frame}{some preface as we're transitioning into a different part of the course, probably} 

Remember when we were talking about computability of \textit{sets of natural numbers} $A \subseteq \mathbb N$?

\vspace{2mm}

Now we are talking about computability of \textit{languages} $L \subseteq \Sigma^*$, which are sets of strings. 

\vspace{2mm}

But really, they're the same thing! Each string has a G*del number. When we talk about a language $L \subseteq \Sigma^*$, we can instead be talking about
$$A = \{e \in \mathbb N: \text{$e$ is the G*dot number for some string in $L$}\}$$
and vice versa.

\end{frame}

\begin{frame}{$P$-reducibility :/} 

\textbf{Task:} What does the $P$ in ``$P$-reducibility'' stand for?
\textbf{Answer:} \sout{Psushi} paulinomial.

\vspace{4mm}

\textbf{Definition:} Let $A, B \subseteq \Sigma^*$ be languages. We say $A \leq_p B$ (read ``$A$ polynomially reduces to $B$'') when there exists a polynomial-time computable $f$ such that
$$x \in A \Leftrightarrow f(x) \in B.$$

\textbf{Fact:}
\begin{quote}
A language is polynomial-time computable on your computer if and only if it is polynomial-time computable by a Turing machine.

\flushright{- Chungus, 2077 (don't quote the chungus on that)}
\end{quote}
\end{frame}

\begin{frame}{$P$-reducibility :/} 

\textbf{Definition:} Let $A, B \subseteq \Sigma^*$ be languages. We say $A \leq_p B$ (read ``$A$ polynomially reduces to $B$'') when there exists a polynomial-time computable $f$ such that
$$x \in A \Leftrightarrow f(x) \in B.$$

\begin{theorem}
If $A \leq_p B$ and $B \leq_p C$, then $A \leq_p C$. 
\end{theorem}

\textbf{Task:} Prove this. 

\end{frame}

\begin{frame}{$P$-reducibility :/} 

\textbf{Definition:} Let $A, B \subseteq \Sigma^*$ be languages. We say $A \leq_p B$ (read ``$A$ polynomially reduces to $B$'') when there exists a polynomial-time computable $f$ such that
$$x \in A \Leftrightarrow f(x) \in B.$$

\begin{theorem}
If $A \leq_p B$ and $B \leq_p C$, then $A \leq_p C$. 
\end{theorem}
\begin{proof}
Since $A \leq_p B$, let $f$ be a polynomial time computable function such that
$x \in A \Leftrightarrow f(x) \in B$. Since $B \leq_p C$, let $g$ be a polynomial time computable function such that $y \in B \Leftrightarrow g(y) \in C$.
Then $g \circ f$ is polynomial-time computable (why?), and
$$x \in A \Leftrightarrow f(x) \in B \Leftrightarrow g(f(x)) \in C.$$
By definition, this means $A \leq_p C$.
\end{proof}

\end{frame}

\begin{frame}{$P$-reducibility :/} 

\textbf{Definition:} Let $A, B \subseteq \Sigma^*$ be languages. We say $A \leq_p B$ (read ``$A$ polynomially reduces to $B$'') when there exists a polynomial-time computable $f$ such that
$$x \in A \Leftrightarrow f(x) \in B.$$

\textbf{Task:} Guess what $A \equiv_p B$ means!

\pause

\textbf{Answer:} $A \equiv_p B$ when $A \leq_p B$ and $B \leq_p A$.

\end{frame}

\begin{frame}{$P$-reducibility :/} 

\textbf{Definition:} Let $A, B \subseteq \Sigma^*$ be languages. We say $A \leq_p B$ (read ``$A$ polynomially reduces to $B$'') when there exists a polynomial-time computable $f$ such that
$$x \in A \Leftrightarrow f(x) \in B.$$

\textbf{Task:} Prove the following theorem, and then try solving $P = NP$ or something:
\begin{theorem}
Let $A, B, C$ be languages. Then
\begin{enumerate}
\item $A \equiv_p A$.
\item $A \equiv_p B$ implies $B \equiv_p A$.
\item $A \equiv_p B$ and $B \equiv_p C$ implies $A \equiv_p C$
\end{enumerate}
So $\equiv_p$ is an equivalence relation on the set of languages.
\end{theorem}
\end{frame}

\begin{frame}{$P$-reducibility :/} 
\begin{theorem}
Let $A, B, C$ be languages. Then
\begin{enumerate}
\item $A \equiv_p A$.
\item $A \equiv_p B$ implies $B \equiv_p A$.
\item $A \equiv_p B$ and $B \equiv_p C$ implies $A \equiv_p C$
\end{enumerate}
So $\equiv_p$ is an equivalence relation on the set of languages.
\end{theorem}

\vspace{-4mm}

\begin{proof}
\begin{enumerate}
\item The function $f(x) = x$ is very polynomial-time computable, and $x \in A \Leftrightarrow f(x) \in A$, so $A \leq_p A$. Bruh. $A \leq_p A$, so $A \equiv_p A$.
\item $A \equiv_p B$ means $A \leq_p B$ and $B \leq_p A$. Bruh. $B \leq_p A$ and $A \leq_p B$ means $B \equiv_p A$.
\item We have $A \leq_p B$, $B \leq_p A$, $B \leq_p C$, and $C \leq_p B$. By the theorem we've proven earlier, $A \leq_p B$ and $B \leq_p C$ imply $A \leq_p C$. Similarly $C \leq_p B$ and $B \leq_p A$ imply $C \leq_p A$. So $A \equiv_p C$.

\end{enumerate}
\end{proof}
\end{frame}

\begin{frame}{Homework cheating time!} 

I'm sure you hate the letter $p$ by now. Why don't we also throw in the letter $P$?

\begin{theorem}
If $A \leq_p B$ and $B \in P$, then $A \in P$.
\end{theorem}

\textbf{Task:} Prove this. Then submit your proof for this under Question 5 of Assignment 3.

(or you know, the proof is actually in the textbook! page 251.)

\end{frame}


\end{document}